\documentclass{beamer}

\mode<presentation>
{
  \usetheme{CambridgeUS}

  \setbeamercovered{transparent}
  % or whatever (possibly just delete it)
}


\usepackage[english]{babel}
\usepackage[utf8]{inputenc}
\usepackage{times}
\usepackage[T1]{fontenc}
\usepackage{amsmath}
\usepackage{minted}
\newminted{py}{gobble=4,mathescape,texcl}

\newcommand{\lra}{\ensuremath{\longrightarrow}}

\title{Advanced Python Constructs}
\subtitle{Generators, decorators and context managers}

\author{Zbigniew Jędrzejewski-Szmek}

\date{EuroSciPy, 2011}

% If you have a file called "university-logo-filename.xxx", where xxx
% is a graphic format that can be processed by latex or pdflatex,
% resp., then you can add a logo as follows:

% \pgfdeclareimage[height=0.5cm]{university-logo}{university-logo-filename}
% \logo{\pgfuseimage{university-logo}}


% If you wish to uncover everything in a step-wise fashion, uncomment
% the following command: 

%\beamerdefaultoverlayspecification{<+->}

\newcommand{\pyver}[1]{\raisebox{0.3em}{\colorbox{yellow}{\tiny #1}}}

\begin{document}

\begin{frame}
  \titlepage
\end{frame}

\begin{frame}{Outline}
  \tableofcontents
  % You might wish to add the option [pausesections]
\end{frame}

\section{Iterators}

\begin{frame}[fragile]{Collections and their iterators}
  \begin{itemize}
    \item
      \fbox{sequence}\pause.\_\_iter\_\_() \pause \lra \fbox{iterator}
      \pause
    \item
      \fbox{iterator}.next() \lra \fbox{item}
      \pause
    \item
      \fbox{iterator}.next() \lra \fbox{item}
      \pause
    \item
      \fbox{iterator}.next() \lra \fbox{item}
  \end{itemize}
\end{frame}

\begin{frame}{Iterators can be non-destructive or destructive}
  \begin{itemize}
    \item \texttt{list}
    \item \texttt{file}
  \end{itemize}
\end{frame}

\begin{frame}[fragile]{How can we create an iterator?}

  \begin{enumerate}[<+->]
    \item write a class with \verb|.__iter__| and \verb|.next|
    \item use a generator expression
    \item write a generator function
  \end{enumerate}
\end{frame}

\begin{frame}[fragile]{1. Iterator class}
  \begin{pycode}
    import random

    class randomaccess(object):
        def __init__(self, list):
            self.indices = range(len(list))
            random.shuffle(self.indices)
            self.list = list
        def next(self):
            return self.list[self.indices.pop()]
  \end{pycode}
\end{frame}

\subsection{Generator expressions}

\begin{frame}[fragile]{2. Generator expressions}
  \begin{pycode}
    # chomped lines
    [line.rstrip() for line in open(some_file)]
  \end{pycode}
  \pause

  \vskip2em
  \begin{pycode}
    # remove comments
    [line for line in open(some_file)
          if not line.startswith('#')]
  \end{pycode}

  \vskip2em
  \begin{minted}{pycon}
>>> type([i for i in 'abc'])
<type 'list'>
>>> type( i for i in 'abc' )
<type 'generator'>
  \end{minted}
\end{frame}

\begin{frame}[fragile]{2. Generator expressions for \texttt{dict} and \texttt{set}}
  \begin{pycode}
    even_squares =
  \end{pycode}
  \onslide<1>
  \begin{pycode}
      set(i**2 for i in range(0, 10, 2))
  \end{pycode}
  \onslide<2->

  \begin{pycode}
      {i**2 for i in range(0, 10, 2)}   # \pyver{(2.7) and (3.1)}
  \end{pycode}

  \vskip3em
  \onslide<3->

  \begin{pycode}
    hex_digits =
  \end{pycode}
  \onslide<3>
  \begin{pycode}
      dict((c,i) for func in (str.lower, str.upper)
                 for i,c in enumerate('abcdef'))
  \end{pycode}

  \onslide<4>
  \begin{pycode}
      {c:i for func in (str.lower, str.upper)
           for i,c in enumerate('abcdef')}     # \pyver{(2.7) and (3.1)}
  \end{pycode}
\end{frame}

\subsection{Generator functions}
\begin{frame}[fragile]{3. Generator functions}
  \begin{columns}[t]
    \column{.5\textwidth}
    \begin{minted}{pycon}
>>> def gen():
...   print '--start'
...   yield 1
...   print '--middle'
...   yield 2
...   print '--stop'
    \end{minted}
    \column{.5\textwidth}
    \pause
    \begin{minted}{pycon}
>>> g = gen()
    \end{minted}
    \vskip-\baselineskip
    \pause
    \vskip0pt
    \begin{minted}{pycon}
>>> g.next()
--start
1
    \end{minted}
    \vskip-\baselineskip
    \pause
    \vskip0pt
    \begin{minted}{pycon}
>>> g.next()
--middle
2
    \end{minted}
    \vskip-\baselineskip
    \pause
    \vskip0pt
    \begin{minted}{pycon}
>>> g.next()
--stop
Traceback (most recent call last):
  ...
StopIteration
    \end{minted}
  \end{columns}
\end{frame}

\begin{frame}[fragile]{Sending information \textbf{to} the generator}
  \begin{columns}[t]
    \column{.5\textwidth}
    \begin{minted}[gobble=6]{python}
      def gen():
        print '--start'
        val = yield 1
        print '--got', val
        print '--middle'
        val = yield 2
        print '--got', val
        print '--done'
    \end{minted}
    \column{.5\textwidth}
    \pause

    \begin{minted}{pycon}
>>> g = gen()
    \end{minted}
    \vskip-\baselineskip
    \pause
    \vskip0pt
    \begin{minted}{pycon}
>>> g.next()
--start
1
    \end{minted}
    \vskip-\baselineskip
    \pause
    \vskip0pt
    \begin{minted}{pycon}
>>> g.send('boo')
--got boo
--middle
2
    \end{minted}
    \vskip-\baselineskip
    \pause
    \vskip0pt
    \begin{minted}{pycon}
>>> g.send('foo')
--got foo
--done
Traceback (most recent call last):
  ...
StopIteration
    \end{minted}
  \end{columns}
\end{frame}

\begin{frame}[fragile]{Grep}
    \begin{minted}[gobble=6]{python}
      def grep(pattern):
          print 'Looking for', pattern
          while True:
              line = yield
              if pattern in line:
                  print line,
    \end{minted}
    \pause
    \begin{minted}{pycon}
>>> g = grep('python')
>>> g.next()
Looking for python
>>> g.send("Yeah, but no, but yeah, but no")
>>> g.send("A series of tubes")
>>> g.send("python generators rock!")
python generators rock!
    \end{minted}
\end{frame}

\subsection{Coroutines}

\begin{frame}{TBD}
  from follow import follow

    %% # Set up a processing pipe : tail -f | grep python
    %% logfile  = open("access-log")
    %% loglines = follow(logfile)
    %% pylines  = grep("python",loglines)

    %% # Pull results out of the processing pipeline
    %% for line in pylines:
    %%     print line,
\end{frame}

\section{Decorators}

\subsection{Function decorators}
\begin{frame}[fragile]{Decorator syntax}
  \begin{columns}[t]
    \column{.4\textwidth}
    \only<2->

    \begin{minted}[gobble=6]{python}
      def deco(orig_f):
          print('decorating:', orig_f)
          return orig_f
    \end{minted}

    \vskip3em

    \begin{minted}[gobble=6]{python}
      @deco
      def func():
          print('in func')
    \end{minted}
    \column{.6\textwidth}

    \only<3->
    \vskip12em
    \begin{minted}[gobble=6]{python}
      def func():
          print('in func')
      func = deco(func)
    \end{minted}
  \end{columns}
\end{frame}

\begin{frame}[fragile]{Set an attribute on the function object}
  \begin{columns}[t]
    \column{0.6\textwidth}
    \begin{minted}[gobble=6]{python}
      class author(object):
          def __init__(self, name):
              self.name = name
          def __call__(self, func):
              func.author = self.name
              return func
    \end{minted}
    \pause
    \vskip1em
    \begin{minted}[gobble=6]{python}
      def author(name):
          def helper(func):
              func.author = name
              return func
          return helper
    \end{minted}
    \column{0.4\textwidth}
    \pause
    \vskip7em
    \begin{minted}{pycon}
>>> @author('Joe')
... def f(): pass
>>> f.author
'Joe'
    \end{minted}
  \end{columns}
\end{frame}

\subsection{Replacing decorators}

\begin{frame}[fragile]{\textbf{Replace} a function}
  \begin{minted}[gobble=4]{python}
    class deprecated(object):
        "Print a deprecation warning once"
        def __init__(self):
            pass
        def __call__(self, func):
            self.func = func
            self.count = 0
            return self.wrapper
        def wrapper(self, *args, **kwargs):
            self.count += 1
            if self.count == 1:
                print self.func.__name__, 'is deprecated'
            return self.func(*args, **kwargs)
  \end{minted}
\end{frame}

\begin{frame}[fragile]{\textbf{Replace} a function}
  \begin{minted}[gobble=4]{python}
    def deprecated(func):
        """Print a deprecation warning once"""
        func.count = 0
        def wrapper(*args, **kwargs):
            func.count += 1
            if func.count == 1:
                print func.__name__, 'is deprecated'
            return func(*args, **kwargs)
        return wrapper
  \end{minted}
  \begin{minted}{pycon}
>>> @deprecated
... def f(): pass
>>> f()
f is deprecated
>>> f()
>>> f()
  \end{minted}
\end{frame}

\begin{frame}[fragile]{Decorators can be stacked}
  \begin{minted}[gobble=4]{python}
    @author('Joe')
    @deprecated
    def func():
        pass
  \end{minted}
  \vskip1em
  \begin{minted}[gobble=4]{python}
    # old style
    def func(): pass
    func = author('Joe')(deprecated(func))
  \end{minted}
\end{frame}

\begin{frame}[fragile]{The docstring problem}
  Our beautiful replacement function lost
  \begin{itemize}
  \item the docstring
  \item attributes
  \item proper argument list
  \end{itemize}

  \pause
  \vskip1em
  \begin{minted}[gobble=4]{python}
    update_wrapper(wrapper, wrapped)
  \end{minted}
  \pause
  \begin{itemize}
    \item \verb+__doc__+
      \pause
    \item \verb+__module__+ and \verb+__name__+
      \pause
    \item \verb+__dict__+
    \item \verb+eval+ is required for the rest :(
  \end{itemize}
\end{frame}

\begin{frame}[fragile]{Replace a function, keep the docstring}
  \begin{minted}[gobble=4]{python}
    def deprecated(func):
        """Print a deprecation warning once"""
        func.count = 0
        def wrapper(*args, **kwargs):
            func.count += 1
            if func.count == 1:
                print func.__name__, 'is deprecated'
            return func(*args, **kwargs)
        return update_wrapper(wrapper, func)
  \end{minted}
\end{frame}

\subsection{Class decorators}

\begin{frame}[fragile]{Add autogenerated \texttt{\_\_init\_\_}}
  \begin{minted}[gobble=4]{python}
    from new import instancemethod

    def autoinit(*allowed_args):
      allowed_args = frozenset(allowed_args)

      def decorate(cls):
        def __init__(self, **kwargs):
          "Set all keyword arguments as attributes"
          bad = set(kwargs.keys()) - allowed_args
          if bad:
            raise TypeError('bad args: '+', '.join(bad))
          self.__dict__.update(kwargs)
        cls.__init__ = instancemethod(__init__, None, cls)
        return cls
      return decorate
  \end{minted}
\end{frame}

\begin{frame}[fragile]{Add autogenerated \texttt{\_\_init\_\_}}
  \begin{minted}{pycon}
>>> @autoinit('foo')
... class A(object):pass
>>> a = A(foo=3)
>>> a.foo
3
>>> A(foo=1, boo=2, goo=3)
Traceback (most recent call last):
  File "<stdin>", line 1, in <module>
  File "<stdin>", line 8, in __init__
TypeError: bad args: goo, boo
  \end{minted}
\end{frame}

\subsection{Examples}

\begin{frame}[fragile]{Decorators in the stdlib}
  \begin{columns}[t]
    \column{0.6\textwidth}
    \pause
    \begin{pycode}
    class A(object):
       def method(self, *args):
          return 1
    \end{pycode}
    \pause
    \begin{pycode}
       @classmethod
       def cmethod(cls, *args):
          return 2
    \end{pycode}
    \pause
    \begin{pycode}
       @staticmethod
       def smethod(*args):
          return 3
    \end{pycode}
    \pause
    \begin{pycode}
       @property
       def notamethod(*args):
          return 4
    \end{pycode}

    \column{0.4\textwidth}
    \only<2->{%
>{>}> a = A()\\
>{>}> a.method()\\
1\\
}
    \only<3->{%
>{>}> a = A()\\
>{>}> a.cmethod()\\
2\\
}
    \only<4->{%
>{>}> a = A()\\
>{>}> a.smethod()\\
3\\
}
    \only<5->{%
>{>}> a = A()\\
>{>}> a.notamethod\\
4\\
}
  \end{columns}
\end{frame}

\begin{frame}[fragile]{The \texttt{property} decorator}
  \begin{pycode}
    class Rectangle(object):
        def __init__(self, edge):
            self.edge = edge

        @property
        def area(self):
            """Computed area.
            """
            return self.edge**2
  \end{pycode}
  \vskip20em
\end{frame}
\begin{frame}[fragile]{The \texttt{property} decorator}
  \begin{pycode}
    class Rectangle(object):
        def __init__(self, edge):
            self.edge = edge

        @property
        def area(self):
            """Computed area.
            Setting this updates the edge length!
            """
            return self.edge**2

        @area.setter
        def area(self, area):
            self.edge = area ** 0.5
  \end{pycode}
  \vskip20em
\end{frame}

\section{Context managers}

\begin{frame}[fragile]{Using a context manager}
  \begin{pycode}
    with manager as var:
        do_something(var)
  \end{pycode}
  \pause
  \vskip1em
  \begin{pycode}
    var = manager.__enter__()
    try:
        do_something(var)
    finally:
        manager.__exit__(None, None, None)
  \end{pycode}
\end{frame}

\begin{frame}[fragile]{Context manager: \texttt{closing}}
  \begin{pycode}
     class closing(object):
         def __init__(self, obj):
             self.obj = obj
         def __enter__(self):
             return self.obj
         def __exit__(self, *args):
             self.obj.close()
  \end{pycode}
  \pause
  \begin{minted}{pycon}
>>> with closing(open('/tmp/file', 'w')) as f:
...   f.write('the contents\n')
  \end{minted}
\end{frame}

\begin{frame}[fragile]{Locking files with \texttt{flock(2)}}
  \begin{pycode}
    def add_user(login, uid, gid, name, home,
                 shell='/bin/bash'):
        fields = (login, str(uid), str(gid),
                  name, home, shell)
        f = open(PASSWD_FILE, 'w')
        fcntl.lockf(f.fileno(), fcntl.LOCK_EX)
        try:
            f.seek(0, io.SEEK_END)
            f.write(':'.join(fields) + '\n')
            f.flush()
        finally:
            fcntl.lockf(f.fileno(), fcntl.LOCK_UN)
  \end{pycode}
\end{frame}

\begin{frame}[fragile]{Context manager: \texttt{flocked}}
  \begin{pycode}
     import fcntl, io

     class flocked(object):
         def __init__(self, file):
             self.file = file
         def __enter__(self):
             fcntl.lockf(self.file.fileno(), fcntl.LOCK_EX)
             return self.file
         def __exit__(self, *args):
             fcntl.lockf(self.file.fileno(), fcntl.LOCK_UN)
  \end{pycode}
\end{frame}

\begin{frame}[fragile]{Locking files with \texttt{flocked}}
  \begin{pycode}
    PASSWD_FILE = '/tmp/passwd'
    def add_user(login, uid, gid, name, home,
                 shell='/bin/bash'):
        fields = (login, str(uid), str(gid),
                  name, home, shell)
        with open(PASSWD_FILE, 'w') as f, flocked(f):
            f.seek(0, io.SEEK_END)
            f.write(':'.join(fields) + '\n')
            f.flush()
  \end{pycode}
\end{frame}

\begin{frame}[fragile]{Context managers in the stdlib}
  \pause
  \begin{itemize}
    \item all file-like objects
      \begin{itemize}
        \item \verb|file|
        \item \verb|fileinput|, \verb|tempfile| \pyver{(3.2)}
        \item \verb|bz2.BZ2File|, \verb|gzip.GzipFile|
          \verb|tarfile.TarFile|, \verb|zipfile.ZipFile|
        \item \verb|ftplib|, \verb|nntplib| \pyver{(3.2 or 3.3)}
      \end{itemize}
    \pause
    \item locks
      \begin{itemize}
      \item \verb|multiprocessing.RLock|
      \item \verb|multiprocessing.Semaphore|
      \item \verb|memoryview| \pyver{(3.2 and 2.7)}
      \end{itemize}
    \pause
    \item \verb|decimal.localcontext|
    \pause
    \item \verb|warnings.catch_warnings|
    \pause
    \item \verb|contextlib.closing|
    \pause
    \item parallel programming
      \begin{itemize}
      \item \verb|concurrent.futures.ThreadPoolExecutor| \pyver{(3.2)}
      \item \verb|concurrent.futures.ProcessPoolExecutor| \pyver{(3.2)}
      \pause
      \item \verb|nogil| \pyver{(cython only)}
      \end{itemize}
  \end{itemize}
\end{frame}

\begin{frame}[fragile]{Managing exceptions}
  \begin{pycode}
    class Manager(object):
        ...
        def __exit__(self, type, value, traceback):
            swallow = True
            return swallow
  \end{pycode}
\end{frame}

\begin{frame}[fragile]{Unittesting thrown exceptions}
  \begin{pycode}
    def test_indexing():
       try:
          {}['foo']
       except KeyError:
          pass
  \end{pycode}
\end{frame}

\begin{frame}[fragile]{Unittesting thrown exceptions}
  \begin{pycode}
    class assert_raises(object):
        def __init__(self, type):
            self.type = type
        def __enter__(self):
            pass
        def __exit__(self, type, value, traceback):
            if type is None:
                raise AssertionError('exception expected')
            if issubclass(type, self.type):
                return True
            raise AssertionError('wrong exception type')
  \end{pycode}
  \pause
  \begin{pycode}
    def test_indexing():
        with assert_raises(KeyError):
            {}['foo']
  \end{pycode}
\end{frame}

\subsection{Generators as context managers}

\begin{frame}[fragile]{Writing context managers as generators}
  \begin{minted}[gobble=4]{python}
    class manager(object):
        def __init__(self, <arguments>):
            ...
        def __enter__(self):
            <setup>
            return <value>
        def __exit__(self, *exc_info):
            <cleanup>
  \end{minted}
  \pause
  \begin{minted}[gobble=4]{python}
    @contextlib.contextmanager
    def some_generator(<arguments>):
        <setup>
        try:
            yield <value>
        finally:
            <cleanup>
  \end{minted}
\end{frame}

\begin{frame}[fragile]{\texttt{assert\_raises} as a function}
  \begin{pycode}
    @contextlib.contextmanager
    def assert_raises(exc):
        try:
            yield
        except exc:
            pass
        except Exception as value:
            raise AssertionError('wrong exception type')
        else:
            raise AssertionError(exc.__name__+' expected')
  \end{pycode}
\end{frame}

\begin{frame}[fragile]{Let's test \texttt{assert\_raises}}
  \begin{minted}{pycon}
>>> with assert_raises(KeyError):
...   pass
Traceback (most recent call last):
  ...
AssertionError: KeyError expected
>>> with assert_raises(KeyError):
...   {}['foo']
...
>>> with assert_raises(KeyError):
...   1/0
Traceback (most recent call last):
  ...
ZeroDivisionError: division by zero

During handling of the above exception, another exception occurred:

Traceback (most recent call last):
  ...
AssertionError: wrong exception type
  \end{minted}
\end{frame}

\begin{frame}[fragile]{Let's test \texttt{assert\_raises}}
  \begin{minted}{pycon}
>>> @assert_raises(KeyError)
... def f(exc):
...   if exc:
...     raise exc
>>> f(None)
Traceback (most recent call last):
  ...
AssertionError: KeyError expected
  \end{minted}
\end{frame}

% documentation ommission: contextlib.contextmanager documentation is too short,
% doesn't mention c.m. being a decorator too

% Structuring a talk is a difficult task and the following structure
% may not be suitable. Here are some rules that apply for this
% solution: 

% - Exactly two or three sections (other than the summary).
% - At *most* three subsections per section.
% - Talk about 30s to 2min per frame. So there should be between about
%   15 and 30 frames, all told.

% - A conference audience is likely to know very little of what you
%   are going to talk about. So *simplify*!
% - In a 20min talk, getting the main ideas across is hard
%   enough. Leave out details, even if it means being less precise than
%   you think necessary.
% - If you omit details that are vital to the proof/implementation,
%   just say so once. Everybody will be happy with that.

\begin{frame}{Make Titles Informative. Use Uppercase Letters.}{Subtitles are optional.}
  % - A title should summarize the slide in an understandable fashion
  %   for anyone how does not follow everything on the slide itself.

  \begin{itemize}
  \item
    Use \texttt{itemize} a lot.
  \item
    Use very short sentences or short phrases.
  \end{itemize}
\end{frame}

\begin{frame}{Make Titles Informative.}

  You can create overlays\dots
  \begin{itemize}
  \item using the \texttt{pause} command:
    \begin{itemize}
    \item
      First item.
      \pause
    \item    
      Second item.
    \end{itemize}
  \item
    using overlay specifications:
    \begin{itemize}
    \item<3->
      First item.
    \item<4->
      Second item.
    \end{itemize}
  \item
    using the general \texttt{uncover} command:
    \begin{itemize}
      \uncover<5->{\item
        First item.}
      \uncover<6->{\item
        Second item.}
    \end{itemize}
  \end{itemize}
\end{frame}


\section*{Summary}

\begin{frame}{Summary}

  \begin{itemize}[<+->]
  \item
    \alert{Generators} make good iterators
  \item
    \alert{Decorators} make wrapping and altering\\functions and classes easy
  \item
    \alert{Context managers} outsource \texttt{except..finally} blocks
  \end{itemize}
\end{frame}


% All of the following is optional and typically not needed. 
\appendix
\section<presentation>*{\appendixname}
\subsection<presentation>*{For Further Reading}

\begin{frame}[allowframebreaks]
  \frametitle<presentation>{For Further Reading}
    
  \begin{thebibliography}{10}
    
  \beamertemplatebookbibitems
  % Start with overview books.

  \bibitem{Author1990}
    A.~Author.
    \newblock {\em Handbook of Everything}.
    \newblock Some Press, 1990.
 
    
  \beamertemplatearticlebibitems
  % Followed by interesting articles. Keep the list short. 

  \bibitem{Someone2000}
    S.~Someone.
    \newblock On this and that.
    \newblock {\em Journal of This and That}, 2(1):50--100,
    2000.
  \end{thebibliography}
\end{frame}

\end{document}
