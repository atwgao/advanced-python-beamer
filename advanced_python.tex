\documentclass{beamer}

\mode<presentation>
{
  \usetheme{CambridgeUS}

  \setbeamercovered{transparent}
  % or whatever (possibly just delete it)
}


\usepackage[english]{babel}
\usepackage[utf8]{inputenc}
\usepackage{times}
\usepackage[T1]{fontenc}
\usepackage{amsmath}
\usepackage{amssymb}
\usepackage{fancybox}
\usepackage{minted}
\newminted{py}{gobble=4,mathescape,texcl}

\newcommand{\lra}{\ensuremath{\longrightarrow}}
\newcommand{\lla}{\ensuremath{\longleftarrow}}

\title{Advanced Python Constructs}
\subtitle{Generators, decorators and context managers}

\author{Zbigniew Jędrzejewski-Szmek}

\setbeamertemplate{footline}
{
  \leavevmode%
  \hbox{%
  \begin{beamercolorbox}[wd=\paperwidth,ht=2.25ex,dp=1ex,right]{date in head/foot}%
    \usebeamerfont{date in head/foot}\insertshortdate{}\hspace*{2em}
    \insertframenumber{} / \inserttotalframenumber\hspace*{2ex}
  \end{beamercolorbox}}%
  \vskip0pt%
}

\date{EuroSciPy, 2011}

% If you have a file called "university-logo-filename.xxx", where xxx
% is a graphic format that can be processed by latex or pdflatex,
% resp., then you can add a logo as follows:

% \pgfdeclareimage[height=0.5cm]{university-logo}{university-logo-filename}
% \logo{\pgfuseimage{university-logo}}


% If you wish to uncover everything in a step-wise fashion, uncomment
% the following command:

%\beamerdefaultoverlayspecification{<+->}

\newcommand{\pyver}[1]{\raisebox{0.3em}{\colorbox{yellow}{\tiny #1}}}

\begin{document}

\begin{frame}
  \titlepage
\end{frame}

\begin{frame}{Outline}
  \tableofcontents
  % You might wish to add the option [pausesections]
\end{frame}

\section{Iterators}

\begin{frame}{}
  \center\Huge Iterators
\end{frame}

\begin{frame}[fragile]{Collections and their iterators}
  \begin{itemize}
    \item
      \fbox{sequence}\pause.\_\_iter\_\_() \pause \lra \fbox{iterator}
      \pause
    \item
      \fbox{iterator}.next() \lra \fbox{item}
      \pause
    \item
      \fbox{iterator}.next() \lra \fbox{item}
      \pause
    \item
      \fbox{iterator}.next() \lra \fbox{item}
  \end{itemize}
\end{frame}

\begin{frame}{Iterators can be non-destructive or destructive}
  \begin{itemize}
    \item \texttt{list}
    \item \texttt{file}
  \end{itemize}
\end{frame}

\begin{frame}[fragile]{How can we create an iterator?}

  \begin{enumerate}[<+->]
    \item write a class with \verb|.__iter__| and \verb|.next|
    \item use a generator expression
    \item write a generator function
  \end{enumerate}
\end{frame}

\subsection{Iterator classes}

\begin{frame}[fragile]{1. Iterator class}
  \begin{pycode}
    import random

    class randomaccess(object):
        def __init__(self, list):
            self.indices = range(len(list))
            random.shuffle(self.indices)
            self.list = list
        def next(self):
            return self.list[self.indices.pop()]
  \end{pycode}
\end{frame}

\subsection{Generator expressions}

\begin{frame}[fragile]{2. Generator expressions}
  \begin{pycode}
    # chomped lines
    [line.rstrip() for line in open(some_file)]
  \end{pycode}
  \pause

  \vskip2em
  \begin{pycode}
    # remove comments
    [line for line in open(some_file)
          if not line.startswith('#')]
  \end{pycode}
  \pause

  \vskip2em
  \begin{minted}{pycon}
>>> type([i for i in 'abc'])
<type 'list'>
>>> type( i for i in 'abc' )
<type 'generator'>
  \end{minted}
\end{frame}

\begin{frame}[fragile]{2. Generator expressions for \texttt{dict} and \texttt{set}}
  \begin{pycode}
    even_squares =
  \end{pycode}
  \onslide<1>
  \begin{pycode}
      set(i**2 for i in range(0, 10, 2))
  \end{pycode}
  \onslide<2->

  \begin{pycode}
      {i**2 for i in range(0, 10, 2)}   # \pyver{(2.7) and (3.1)}
  \end{pycode}

  \vskip3em
  \onslide<3->

  \begin{pycode}
    hex_digits =
  \end{pycode}
  \onslide<3>
  \begin{pycode}
      dict((c,i) for func in (str.lower, str.upper)
                 for i,c in enumerate('abcdef'))
  \end{pycode}

  \onslide<4>
  \begin{pycode}
      {c:i for func in (str.lower, str.upper)
           for i,c in enumerate('abcdef')}     # \pyver{(2.7) and (3.1)}
  \end{pycode}
\end{frame}

\subsection{Generator functions}
\begin{frame}[fragile]{3. Generator functions}
  \begin{columns}[t]
    \column{.5\textwidth}
    \begin{minted}{pycon}
>>> def gen():
...   print '--start'
...   yield 1
...   print '--middle'
...   yield 2
...   print '--stop'
    \end{minted}
    \column{.5\textwidth}
    \pause
    \begin{minted}{pycon}
>>> g = gen()
    \end{minted}
    \vskip-\baselineskip
    \pause
    \vskip0pt
    \begin{minted}{pycon}
>>> g.next()
--start
1
    \end{minted}
    \vskip-\baselineskip
    \pause
    \vskip0pt
    \begin{minted}{pycon}
>>> g.next()
--middle
2
    \end{minted}
    \vskip-\baselineskip
    \pause
    \vskip0pt
    \begin{minted}{pycon}
>>> g.next()
--stop
Traceback (most recent call last):
  ...
StopIteration
    \end{minted}
  \end{columns}
\end{frame}

\begin{frame}[fragile]{Python version of Unix 'tail -f'}
  \vskip5em
  \begin{pycode}
    logfile = open('/var/log/messages')
    for line in follow(logfile):
        print line,
  \end{pycode}

  \vskip5em
  \color{darkgray}{http://www.dabeaz.com/coroutines/}
\end{frame}

\begin{frame}[fragile]{Inotify}
  \begin{minted}{pycon}
>>> watcher = inotify.watcher.Watcher()
>>> watcher.add('/tmp/file', inotify.IN_MODIFY)
1
>>> watcher.read()
[Event(path='/tmp/file', wd=1, mask=IN_MODIFY)]
  \end{minted}

  \pause
  \texttt{IN\_MODIFY}\\
  \texttt{IN\_MOVE}\\
  \texttt{IN\_CLOSE\_NOWRITE}\\
  \texttt{IN\_CLOSE\_WRITE}\\
  ...
\end{frame}

\begin{frame}[fragile]{Implementation of our 'tail -f'}
\begin{minted}{python}
import os
# https://bitbucket.org/bos/python-inotify
import inotify, inotify.watcher

def follow(file):
    watcher = inotify.watcher.Watcher()
    watcher.add(file.name, inotify.IN_MODIFY)
    file.seek(0, os.SEEK_END)
    while True:
        line = file.readline()
        if not line:
            watcher.read()
            continue
        yield line
\end{minted}
\end{frame}

\begin{frame}{Generators as pipelines}
  \mbox{\alert<6>{\texttt{for x in s:}}}%
  \lla \ovalbox{\alert<5>{generator}}%
  \lla \ovalbox{\alert<4>{generator}}%
  \lla \ovalbox{\alert<3>{generator}}%
  \lla \parbox{4.5em}{\alert<2>{input\\sequence}}
\end{frame}

\begin{frame}[fragile]{A pipeline example}
  \begin{pycode}
    def grep(pattern, source):
        for line in source:
            if pattern in line:
                 yield line
  \end{pycode}
  \pause

  \begin{pycode}
    # our pipeline:
    #     tail -f /var/log/messages | grep python
    logfile = open('/var/log/messages')

    lines = follow(logfile)
    interesting = grep('python', lines)

    for line in interesting:
        print line,
  \end{pycode}
\end{frame}

\begin{frame}[fragile]{Yield as an expression}
  \begin{pycode}
    def gen():
        val = yield
  \end{pycode}

  \vskip5em
  \pause
  Value is send when \texttt{gen().send(val)} is used, not \texttt{gen().next()}
\end{frame}

\begin{frame}[fragile]{Sending information \textbf{to} the generator}
  \begin{columns}[t]
    \column{.5\textwidth}
    \begin{minted}[gobble=6]{python}
      def gen():
        print '--start'
        val = yield 1
        print '--got', val
        print '--middle'
        val = yield 2
        print '--got', val
        print '--done'
    \end{minted}
    \column{.5\textwidth}
    \pause

    \begin{minted}{pycon}
>>> g = gen()
    \end{minted}
    \vskip-\baselineskip
    \pause
    \vskip0pt
    \begin{minted}{pycon}
>>> g.next()
--start
1
    \end{minted}
    \vskip-\baselineskip
    \pause
    \vskip0pt
    \begin{minted}{pycon}
>>> g.send('boo')
--got boo
--middle
2
    \end{minted}
    \vskip-\baselineskip
    \pause
    \vskip0pt
    \begin{minted}{pycon}
>>> g.send('foo')
--got foo
--done
Traceback (most recent call last):
  ...
StopIteration
    \end{minted}
  \end{columns}
\end{frame}

\begin{frame}[fragile]{Grep}
%  \thinspace\verb|# grep pattern file|\thinspace
%  \pause

  \begin{minted}[gobble=4]{python}
    def grep(pattern):
        print 'Looking for', pattern
        while True:
            line = yield
            if pattern in line:
                print line,
  \end{minted}
  \pause
  \begin{minted}{pycon}
>>> g = grep('python')
    \end{minted}
    \vskip-\baselineskip
    \pause
    \vskip0pt
    \begin{minted}{pycon}
>>> g.next()
Looking for python
    \end{minted}
    \vskip-\baselineskip
    \pause
    \vskip0pt
    \begin{minted}{pycon}
>>> g.send("Yeah, but no, but yeah, but no")
    \end{minted}
    \vskip-\baselineskip
    \pause
    \vskip0pt
    \begin{minted}{pycon}
>>> g.send("A series of tubes")
    \end{minted}
    \vskip-\baselineskip
    \pause
    \vskip0pt
    \begin{minted}{pycon}
>>> g.send("python generators rock!")
python generators rock!
  \end{minted}
\end{frame}

\subsection{Coroutines}

\begin{frame}{Coroutines}
  Let's push the data in other direction!
  \vskip4em

  \mbox{\alert<2>{\texttt{for x in s:}}}%
  \lra \ovalbox{\alert<3>{coroutine}}%
  \lra \ovalbox{\alert<4>{coroutine}}%
  \lra \ovalbox{\alert<5>{coroutine}}%
  \lra \ovalbox{\alert<6>{a sink}}
  \vskip2em
  \only<7->{or}
  \vskip2em
  \only<8>{%
    \ovalbox{source}%
    \lra \ovalbox{coroutine}%
    \lra \ovalbox{coroutine}%
    \lra \ovalbox{coroutine}%
    \lra \ovalbox{a sink}}
\end{frame}

\begin{frame}[fragile]{\texttt{follow} as a coroutine source}
\begin{semiverbatim}
import os
import inotify, inotify.watcher

def follow(file\alert<2>{, target}):
    watcher = inotify.watcher.Watcher()
    watcher.add(file.name, inotify.IN_MODIFY)
    file.seek(0, os.SEEK_END)
    while True:
        line = file.readline()
        if not line:
            watcher.read()
            continue
        \alert<2>{target.send(line)}
\end{semiverbatim}
\end{frame}

\begin{frame}[fragile]{\texttt{grep} as a coroutine}
  \begin{semiverbatim}
    def grep(pattern\alert<2>{, target}):
        print 'Looking for', pattern
        while True:
            \alert<2>{line = yield}
            if pattern in line:
                \alert<2>{target.send(line)}
  \end{semiverbatim}
\end{frame}

\begin{frame}[fragile]{A sink: \texttt{printer}}
  \begin{pycode}
    def printer():
        while True:
            line = yield
            print line,
  \end{pycode}
\end{frame}

\begin{frame}[fragile]{A coroutine pipeline}
  \begin{block}{our pipeline}
    tail -f /var/log/messages | grep python
  \end{block}
  \vskip\baselineskip
  \begin{pycode}
    sink = printer()
    next(sink)
  \end{pycode}
  \vskip-\baselineskip
  \pause
  \vskip0pt
  \begin{pycode}

    interesting = grep('python', sink)
    next(interesting)
  \end{pycode}
  \vskip-\baselineskip
  \pause
  \vskip0pt
  \begin{pycode}

    logfile = open('/var/log/messages')
    follow(logfile, interesting)
  \end{pycode}

  \pause
  \vskip2em
  \color{darkgray}{http://www.dabeaz.com/coroutines/}
\end{frame}

\begin{frame}[fragile]{Throwing exceptions \textbf{into} the generator}
  \begin{minted}{pycon}
>>> def f():
...   yield
>>> g = f()
    \end{minted}
    \vskip-\baselineskip
    \pause
    \vskip0pt
    \begin{minted}{pycon}
>>> g.next()
    \end{minted}
    \vskip-\baselineskip
    \pause
    \vskip0pt
    \begin{minted}{pycon}
>>> g.throw(IndexError)
Traceback (most recent call last):
  File "<stdin>", line 1, in <module>
  File "<stdin>", line 2, in f
IndexError
  \end{minted}
\end{frame}

\begin{frame}[fragile]{Destroying generators}
  \texttt{.close()} is used to destroy resources tied up in the generator

  \pause
  \vskip1em
  \begin{minted}{pycon}
>>> def f():
...   try:
...     yield
...   except GeneratorExit:
...     print "bye!"
>>> g = f()
>>> g.next()
>>> g.close()
bye!
  \end{minted}
\end{frame}

\section{Decorators}

\begin{frame}{}
  \center\Huge Decorators
\end{frame}

\subsection{Function decorators}

\begin{frame}[fragile]{Decorator syntax}
  \begin{columns}[t]
    \column{.5\textwidth}
    \begin{minted}[gobble=6]{python}
      @deco
      def func():
          print 'in func'
    \end{minted}
    \column{.5\textwidth}

    \pause
    \begin{minted}[gobble=6]{python}
      def func():
          print 'in func'
      func = deco(func)
    \end{minted}
  \end{columns}

  \pause
  \vskip5em
    \begin{minted}[gobble=6]{python}
      def deco(orig_f):
          print 'decorating:', orig_f
          return orig_f
    \end{minted}
\end{frame}

\begin{frame}[fragile]{Set an attribute on the function object}
  \begin{columns}[t]
    \column{0.4\textwidth}
    \vskip7em
    \begin{minted}{pycon}
>>> @author('Joe')
... def f():
...   pass
>>> f.author
'Joe'
    \end{minted}
    \column{0.6\textwidth}
    \pause
    \begin{minted}[gobble=6]{python}
      class author(object):
          def __init__(self, name):
              self.name = name
          def __call__(self, func):
              func.author = self.name
              return func
    \end{minted}
    \pause
    \vskip1em
    \begin{minted}[gobble=6]{python}
      def author(name):
          def helper(func):
              func.author = name
              return func
          return helper
    \end{minted}
  \end{columns}
\end{frame}

\subsection{Wrapping decorators}

\begin{frame}[fragile]{\textbf{Replace} a function}
  \begin{minted}[gobble=4]{python}
    class deprecated(object):
        "Print a deprecation warning once"
        def __init__(self):
            pass
        def __call__(self, func):
            self.func = func
            self.count = 0
            return self.wrapper
        def wrapper(self, *args, **kwargs):
            self.count += 1
            if self.count == 1:
                print self.func.__name__, 'is deprecated'
            return self.func(*args, **kwargs)
  \end{minted}
\end{frame}

\begin{frame}[fragile]{\textbf{Replace} a function}
  \begin{minted}[gobble=4]{python}
    def deprecated(func):
        """Print a deprecation warning once"""
        func.count = 0
        def wrapper(*args, **kwargs):
            func.count += 1
            if func.count == 1:
                print func.__name__, 'is deprecated'
            return func(*args, **kwargs)
        return wrapper
  \end{minted}

  \pause
  \begin{minted}{pycon}
>>> @deprecated
... def f(): pass
>>> f()
f is deprecated
>>> f()
>>> f()
  \end{minted}
\end{frame}

\begin{frame}[fragile]{Decorators can be stacked}
  \begin{minted}[gobble=4]{python}
    @author('Joe')
    @deprecated
    def func():
        pass
  \end{minted}
  \vskip1em
  \begin{minted}[gobble=4]{python}
    # old style
    def func(): pass
    func = author('Joe')(deprecated(func))
  \end{minted}
\end{frame}

\begin{frame}[fragile]{The docstring problem}
  Our beautiful replacement function lost
  \begin{itemize}
  \item the docstring
  \item attributes
  \item proper argument list
  \end{itemize}

  \pause
  \vskip1em
  \begin{minted}[gobble=4]{python}
    update_wrapper(wrapper, wrapped)
  \end{minted}
  \pause
  \begin{itemize}
    \item \verb+__doc__+
      \pause
    \item \verb+__module__+ and \verb+__name__+
      \pause
    \item \verb+__dict__+
      \pause
    \item \verb+eval+ is required for the rest :(
    \item module \verb+decorator+ compiles functions dynamically
  \end{itemize}
\end{frame}

\begin{frame}[fragile]{Replace a function, keep the docstring}
  \begin{minted}[gobble=4]{python}
    import functools

    def deprecated(func):
        """Print a deprecation warning once"""
        func.count = 0
        def wrapper(*args, **kwargs):
            func.count += 1
            if func.count == 1:
                print func.__name__, 'is deprecated'
            return func(*args, **kwargs)
        return functools.update_wrapper(wrapper, func)
  \end{minted}

  \vskip1.5em
  \only<2>{\color{red}{pickle!}}
\end{frame}

\subsection{Class decorators}

\begin{frame}[fragile]{Decorators work for classes too}
  \begin{pycode}
    @deco
    class A(object):
        pass
  \end{pycode}
\end{frame}

\begin{frame}[fragile]{Add autogenerated \texttt{\_\_init\_\_}}
  \begin{minted}{pycon}
>>> @autoinit('foo')
... class A(object):
...   pass
    \end{minted}
    \vskip-\baselineskip
    \pause
    \vskip0pt
    \begin{minted}{pycon}
>>> a = A(foo=3)
    \end{minted}
    \vskip-\baselineskip
    \pause
    \vskip0pt
    \begin{minted}{pycon}
>>> a.foo
3
    \end{minted}
    \vskip-\baselineskip
    \pause
    \vskip0pt
    \begin{minted}{pycon}
>>> A(foo=1, boo=2, goo=3)
Traceback (most recent call last):
  File "<stdin>", line 1, in <module>
  File "<stdin>", line 8, in __init__
TypeError: bad args: goo, boo
  \end{minted}
\end{frame}

\begin{frame}[fragile]{Add autogenerated \texttt{\_\_init\_\_}}
  \begin{pycode}
    from new import instancemethod

  \end{pycode}
  \pause
  \vskip0pt
  \begin{pycode}
    def autoinit(*allowed_args):
      allowed_args = frozenset(allowed_args)
  \end{pycode}
  \vskip-\baselineskip
  \pause
  \vskip0pt
  \begin{pycode}

      def decorate(cls):
        def __init__(self, **kwargs):
          "Set all keyword arguments as attributes"
          bad = set(kwargs.keys()) - allowed_args
          if bad:
            raise TypeError('bad args: '+', '.join(bad))
          self.__dict__.update(kwargs)
  \end{pycode}
  \vskip-\baselineskip
  \pause
  \vskip0pt
  \begin{pycode}
        cls.__init__ = instancemethod(__init__, None, cls)
        return cls
  \end{pycode}
  \vskip-\baselineskip
  \pause
  \vskip0pt
  \begin{pycode}
      return decorate
  \end{pycode}
\end{frame}

\subsection{Examples}

\begin{frame}[fragile]{Decorators in the stdlib}
  \begin{columns}[t]
    \column{0.6\textwidth}
    \pause
    \begin{pycode}
    class A(object):
       def method(self, *args):
          return 1
    \end{pycode}
    \pause
    \begin{pycode}
       @classmethod
       def cmethod(cls, *args):
          return 2
    \end{pycode}
    \pause
    \begin{pycode}
       @staticmethod
       def smethod(*args):
          return 3
    \end{pycode}
    \pause
    \begin{pycode}
       @property
       def notamethod(*args):
          return 4
    \end{pycode}

    \column{0.4\textwidth}
    \only<2->{%
>{>}> a = A()\\
>{>}> a.method()\\
1\\
}
    \only<3->{%
>{>}> a.cmethod()\\
2\\
>{>}> A.cmethod()\\
2\\
}
    \only<4->{%
>{>}> a.smethod()\\
3\\
>{>}> A.smethod()\\
3\\
}
    \only<5->{%
>{>}> a.notamethod\\
4\\
}
  \end{columns}
\end{frame}

\begin{frame}[fragile]{The \texttt{property} decorator}
  \begin{pycode}
    class Rectangle(object):
        def __init__(self, edge):
            self.edge = edge

        @property
        def area(self):
            """Computed area.
            """
            return self.edge**2
  \end{pycode}
  \vskip20em
\end{frame}
\begin{frame}[fragile]{The \texttt{property} decorator}
  \begin{pycode}
    class Rectangle(object):
        def __init__(self, edge):
            self.edge = edge

        @property
        def area(self):
            """Computed area.
            Setting this updates the edge length!
            """
            return self.edge**2

        @area.setter
        def area(self, area):
            self.edge = area ** 0.5
  \end{pycode}
  \vskip20em
\end{frame}
\begin{frame}[fragile]{The \texttt{property} triple: setter, getter, deleter}
  \begin{itemize}[<+->]
  \item attribute access \verb+a.edge+ calls \verb+area.getx+
    \begin{itemize}[<+->]
    \item set with \verb+@property+
    \end{itemize}
  \item attribute setting \verb+a.edge=3+ calls \verb+area.setx+
    \begin{itemize}[<+->]
    \item set with \verb+.setter+
    \end{itemize}
  \item attribute setting \verb+del a.edge+ calls \verb+area.delx+
    \begin{itemize}[<+->]
    \item set with \verb+.deleter+
    \end{itemize}
  \end{itemize}
\end{frame}

\subsection{Overloading functions}

\begin{frame}[fragile]{Annotations}
  \begin{minted}{pycon}
>>> def func(a: 'comment', b: str) -> 'something':
...    pass
  \end{minted}
  \vskip-\baselineskip
  \pause
  \vskip0pt
  \begin{minted}{pycon}
>>> func.__annotations__
{'a': 'comment',
 'b': <class 'str'>,
 'return': 'something}
  \end{minted}
  \pause
  \begin{minted}{pycon}
>>> import sys
>>> sys.version
'3.2 (r32:88445, Feb 20 2011, 19:50:20) \n[GCC 4.4.5]'
  \end{minted}
\end{frame}

\begin{frame}[fragile]{Overloading functions}
  \begin{pycode}
    class A(object):
        @overloaded
        def __init__(self, number: int):
            print('A.__init__ int', self, number)
            self.number = number
  \end{pycode}
  \pause
  \begin{pycode}
        @__init__.overload
        def __init__(self, string: str):
            print('A.__init__ str', self, string)
            self.__init__(int(string))
  \end{pycode}
\end{frame}

\begin{frame}[fragile]{\texttt{overloaded} decorator}
  \begin{minted}{python}
def overloaded(primary):
    funcs = [primary]
  \end{minted}
  \pause
  \begin{minted}{python}
    def wrapper(self, *args, **kwargs):
        print('overloaded:wrapper', args, kwargs)
        for candidate in funcs:
          print('considering', candidate)
          if _fits(args, candidate):
            return candidate(self, *args, **kwargs)
        raise ValueError('no fitting method')
  \end{minted}
  \pause
  \begin{minted}{python}
    def overload(func):
        funcs.append(func)
        return wrapper
  \end{minted}
  \pause
  \begin{minted}{python}
    wrapper.overload = overload
    return functools.update_wrapper(wrapper, primary)
  \end{minted}
\end{frame}

\begin{frame}[fragile]{\texttt{overloaded} decorator --- \texttt{\_fits}}
  \begin{minted}{python}
def _fits(args, func):
    argnames = func.__code__.co_varnames[1:]
    types = [func.__annotations__.get(name, object)
             for name in argnames]
    if len(args) != len(types):
        return False
    return all(isinstance(arg, type)
               for arg,type in zip(args,types))
  \end{minted}
\end{frame}

\begin{frame}[fragile]{Testing the overloaded functions}
  \begin{minted}{pycon}
>>> a = A(11)
>>> a.number
11
  \end{minted}
  \vskip-\baselineskip
  \pause
  \vskip0pt
  \begin{minted}{pycon}
>>> b = A('11')
>>> b.number
11
  \end{minted}
  \vskip-\baselineskip
  \pause
  \vskip0pt
  \begin{minted}{pycon}
>>> A(11.0)
Traceback (most recent call last):
  ...
ValueError: no fitting method
  \end{minted}
\end{frame}

\section{Context managers}

\begin{frame}{}
  \center\Huge Context managers
\end{frame}

\begin{frame}[fragile]{Using a context manager}
  \begin{pycode}
    with manager as var:
        do_something(var)
  \end{pycode}
  \pause
  \vskip1em
  \begin{pycode}
    var = manager.__enter__()
    try:
        do_something(var)
    finally:
        manager.__exit__(None, None, None)
  \end{pycode}
\end{frame}

\begin{frame}[fragile]{Context manager: \texttt{closing}}
  \begin{pycode}
     class closing(object):
         def __init__(self, obj):
             self.obj = obj
         def __enter__(self):
             return self.obj
         def __exit__(self, *args):
             self.obj.close()
  \end{pycode}
  \pause
  \begin{minted}{pycon}
>>> with closing(open('/tmp/file', 'w')) as f:
...   f.write('the contents\n')
  \end{minted}
\end{frame}

\begin{frame}[fragile]{\texttt{file} is a context manager}
  \begin{minted}{pycon}
>>> help(file.__enter__)
Help on method_descriptor:

__enter__(...)
    __enter__() -> self.
>>> help(file.__exit__)
Help on method_descriptor:

__exit__(...)
    __exit__(*excinfo) -> None.  Closes the file.
  \end{minted}
  \pause
  \begin{minted}{pycon}
>>> with open('/tmp/file', 'a') as f:
...   f.write('the contents\n')
  \end{minted}
\end{frame}

\begin{frame}[fragile]{Locking files with \texttt{flock(2)}}
  \begin{pycode}
    PASSWD_FILE = '/etc/passwd'

    def add_user(login, uid, gid, name, home,
                 shell='/bin/bash'):
        fields = (login, str(uid), str(gid),
                  name, home, shell)
        f = open(PASSWD_FILE, 'w')
        fcntl.lockf(f.fileno(), fcntl.LOCK_EX)
        try:
            f.seek(0, os.SEEK_END)
            f.write(':'.join(fields) + '\n')
            f.flush()
        finally:
            fcntl.lockf(f.fileno(), fcntl.LOCK_UN)
  \end{pycode}
\end{frame}

\begin{frame}[fragile]{Context manager: \texttt{flocked}}
  \begin{pycode}
    import fcntl

    class flocked(object):
      def __init__(self, file):
         self.file = file
      def __enter__(self):
         fcntl.lockf(self.file.fileno(), fcntl.LOCK_EX)
         return self.file
      def __exit__(self, *args):
         fcntl.lockf(self.file.fileno(), fcntl.LOCK_UN)
  \end{pycode}
\end{frame}

\begin{frame}[fragile]{Locking files with \texttt{flocked}}
  \begin{pycode}
    PASSWD_FILE = '/etc/passwd'

    def add_user(login, uid, gid, name, home,
                 shell='/bin/bash'):
        fields = (login, str(uid), str(gid),
                  name, home, shell)
        with open(PASSWD_FILE, 'w') as f, flocked(f):
            f.seek(0, io.SEEK_END)
            f.write(':'.join(fields) + '\n')
            f.flush()
  \end{pycode}
\end{frame}

\begin{frame}[fragile]{Context managers in the stdlib}
  \pause
  \begin{itemize}
    \item all file-like objects
      \begin{itemize}
        \item \verb|file|
        \item \verb|fileinput|, \verb|tempfile| \pyver{(3.2)}
        \item \verb|bz2.BZ2File|, \verb|gzip.GzipFile|
          \verb|tarfile.TarFile|, \verb|zipfile.ZipFile|
        \item \verb|ftplib|, \verb|nntplib| \pyver{(3.2 or 3.3)}
      \end{itemize}
    \pause
    \item locks
      \begin{itemize}
      \item \verb|multiprocessing.RLock|
      \item \verb|multiprocessing.Semaphore|
      \end{itemize}
    \pause
    \item \verb|memoryview| \pyver{(3.2 and 2.7)}
    \pause
    \item \verb|decimal.localcontext|
    \pause
    \item \verb|warnings.catch_warnings|
    \pause
    \item \verb|contextlib.closing|
    \pause
    \item parallel programming
      \begin{itemize}
      \item \verb|concurrent.futures.ThreadPoolExecutor| \pyver{(3.2)}
      \item \verb|concurrent.futures.ProcessPoolExecutor| \pyver{(3.2)}
      \pause
      \item \verb|nogil| \pyver{(cython only)}
      \end{itemize}
  \end{itemize}
\end{frame}

\begin{frame}[fragile]{Managing exceptions}
  \begin{pycode}
    class Manager(object):
        ...
        def __exit__(self, type, value, traceback):
            swallow = True
            return swallow
  \end{pycode}
\end{frame}

\begin{frame}[fragile]{Unittesting thrown exceptions}
  \begin{pycode}
    def test_indexing():
       try:
          {}['foo']
       except KeyError:
          pass
  \end{pycode}
\end{frame}

\begin{frame}[fragile]{Unittesting thrown exceptions}
  \begin{pycode}
    class assert_raises(object):
        def __init__(self, type):
            self.type = type
        def __enter__(self):
            pass
        def __exit__(self, type, value, traceback):
            if type is None:
                raise AssertionError('exception expected')
            if issubclass(type, self.type):
                return True
            raise AssertionError('wrong exception type')
  \end{pycode}
  \pause
  \begin{pycode}
    def test_indexing():
        with assert_raises(KeyError):
            {}['foo']
  \end{pycode}
\end{frame}

\subsection{Generators as context managers}

\begin{frame}[fragile]{Writing context managers as generators}
  \begin{pycode}
    @contextlib.contextmanager
    def some_generator(<arguments>):
        <setup>
        try:
            yield <value>
        finally:
            <cleanup>
  \end{pycode}
  \pause
  \begin{pycode}
    class Manager(object):
        def __init__(self, <arguments>):
            ...
        def __enter__(self):
            <setup>
            return <value>
        def __exit__(self, *exc_info):
            <cleanup>
  \end{pycode}
\end{frame}

\begin{frame}[fragile]{\texttt{assert\_raises} as a function}
  \begin{pycode}
    @contextlib.contextmanager
    def assert_raises(exc):
        try:
            yield
        except exc:
            pass
        except Exception as value:
            raise AssertionError('wrong exception type')
        else:
            raise AssertionError(exc.__name__+' expected')
  \end{pycode}
\end{frame}

\begin{frame}[fragile]{Let's test \texttt{assert\_raises}}
  \begin{minted}{pycon}
>>> with assert_raises(KeyError):
...   pass
Traceback (most recent call last):
  ...
AssertionError: KeyError expected

>>> with assert_raises(KeyError):
...   {}['foo']
  \end{minted}
\end{frame}
\begin{frame}[fragile]{Let's test \texttt{assert\_raises}}
  \begin{minted}{pycon}
>>> with assert_raises(KeyError):
...   1/0
Traceback (most recent call last):
  ...
ZeroDivisionError: division by zero

During handling of the above exception, another exception occurred:

Traceback (most recent call last):
  ...
AssertionError: wrong exception type
  \end{minted}
\end{frame}

\begin{frame}[fragile]{Let's test \texttt{assert\_raises}}
  \begin{minted}{pycon}
>>> @assert_raises(KeyError)
... def f(exc):
...   if exc:
...     raise exc
>>> f(None)
Traceback (most recent call last):
  ...
AssertionError: KeyError expected
  \end{minted}
\end{frame}

% documentation ommission: contextlib.contextmanager documentation is too short,
% doesn't mention c.m. being a decorator too

\section*{Summary}

\begin{frame}{Summary}

  \begin{itemize}[<+->]
  \item
    \alert{Generators} make good iterators
  \item
    \alert{Decorators} make wrapping and altering\\functions and classes easy
  \item
    \alert{Context managers} outsource \texttt{except..finally} blocks
  \end{itemize}
\end{frame}


%% % All of the following is optional and typically not needed.
%% \appendix
%% \section<presentation>*{\appendixname}
%% \subsection<presentation>*{For Further Reading}

%% \begin{frame}[allowframebreaks]
%%   \frametitle<presentation>{For Further Reading}

%%   \begin{thebibliography}{10}

%%   \beamertemplatebookbibitems
%%   % Start with overview books.

%%   \bibitem{Author1990}
%%     A.~Author.
%%     \newblock {\em Handbook of Everything}.
%%     \newblock Some Press, 1990.


%%   \beamertemplatearticlebibitems
%%   % Followed by interesting articles. Keep the list short.

%%   \bibitem{Someone2000}
%%     S.~Someone.
%%     \newblock On this and that.
%%     \newblock {\em Journal of This and That}, 2(1):50--100,
%%     2000.
%%   \end{thebibliography}
%% \end{frame}

\end{document}
